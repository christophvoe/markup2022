% Settings for the whole document
\documentclass[aspectratio=169]{beamer}
\usetheme{default}
\usecolortheme{beaver}
\usepackage{amsmath}

% Specifying title dia information
\title{Example document recreation with beamer in \LaTeX}
\author{Nina van Gerwen}
\date{\vspace{2cm} \\ FALL 2022 \\
Markup Language and Reproducible Programming in Statistics}

% Beginning the document
\begin{document}

% Creating the title dia
\maketitle

% First dia with information
\begin{frame}{Outline}

Working with equations \\
	\hspace{0.5cm} Aligning the same equations \\
	\hspace{0.5cm} Omit equation numbering \\
	\hspace{0.5cm} Ugly alignment \\
\vspace{1cm}
Discussion
\indent

\end{frame}

% Second dia with information
\begin{frame}{Working with equations}

We define a set of equations as
	\begin{equation}
		a = b + c^2,
	\end{equation}
	\begin{equation}
		a - c^2 = b,
	\end{equation}
	\begin{equation}
		\text{left side} = \text{right side},
	\end{equation}
	\begin{equation}
		\text{left side} + \text{something} \geq \text{right side},
	\end{equation}
for all something $>$ 0.

\end{frame}

% Third dia with information
\begin{frame}{Aligning the same equations}
Aligning the equations by the equal sign gives a much better view into the placements of the separate equation components.

	\begin{align}
		a &= b + c^2, \\
		a - c^2 &= b, \\
		\text{left side} &= \text{right side}, \\
		\text{left side} + \text{something} &\geq \text{right side},
	\end{align}

\end{frame}

%Fourth dia with information
\begin{frame}{Omit equation numbering}
Alternatively, the equation numbering can be omitted.

	\begin{align*}
		a &= b + c^2, \\
		a - c^2 &= b, \\
		\text{left side} &= \text{right side}, \\
		\text{left side} + \text{something} &\geq \text{right side},
	\end{align*}
	
\end{frame}

%Fifth dia with information
\begin{frame}{Ugly alignment}
Some components do not look well, when aligned. Especially equations with different heights and spacing. For example,

\begin{align}
	E &= mc^2, \\
	m &= \frac{E}{c^2}, \\
	c & = \sqrt{\frac{E}{m}}.
\end{align}

Take that into account.

\end{frame}

%Final dia with information
\begin{frame}{Discussion}
This is where you'd normally give your audience a recap of your talk, where you could discuss e.g. the following
	\begin{itemize}
		\item Your main findings
		\item The consequences of your main findings
		\item Things to do
		\item Any other business not currently investigated, but related to your talk
	\end{itemize}
\end{frame}


\end{document}